\lecture{12}{2022-02-14}{}

\section{Representation}

\begin{defn}{Representation}{}
  A \emph{representation} of a group \(G\) is a homomorphism \(\rho\colon G \to GL(V)\) for some (finite-dimensional) vector space \(V\). The \emph{degree/dimension} of \(\rho\) is the dimension of \(V\).
\end{defn}

From now on, if \(S\) is a set with \(n\) elements, we will identify each element of \(S\) with an element of \(\{1, 2, \dots, n\}\), and therefore we will identify \(\Perm(S)\) with \(S_n\).
Whenever useful, we may choose the ordering of \(S\).

\begin{prop}{}{action-representation}
  Define \(\psi\colon S_n \to GL_n(\mathbb{R})\) by mapping \(\pi\) to a matrix \(M = [a_{ij}]_{n \times n}\) with \[
    a_{ij} =
	\begin{cases}
	  1 & \text{if }j = \pi(i)\\
	  0 & \text{otherwise.}
	\end{cases}
  \] 

  Therefore, the map \[
    G \to \Perm(S) \to S_n \to GL_n(\mathbb{R}),
  \] that sends \[
    g \mapsto \pi_g \to \pi \mapsto M,
  \] 
  is a representation.
\end{prop}
