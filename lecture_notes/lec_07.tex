\lecture{7}{2022-02-02}{Rotation symmetries of a cube}

\subsection*{Application}

\begin{que}{}{}
  What are all rotational symmetries of a cube?
\end{que}

Let \(G\) be the group of rotations of the cube.
Let \(S\) be the set of (the six) faces of the cube.
Let \(G\) act on \(S\) naturally.
Note that this action is transitive, i.e., you can send any face to any other face via rotations.
Also, if we fix \(s \in S\) to be a face, then there are only \(4\) rotations of the cube that fix \(s\).

Therefore, using the \nameref{thm:orbit-stabilizer}, we conclude that \[
  |G| = |\mathcal{O}_s| |G_s| = 6 \cdot 4 = 24.
\] 

Now, if we want to describe all rotational symmetries of a cube and be sure there is no sneaky rotation that we didn't account for, it suffices to describe \(24\) distinct rotations. Here they are:
\begin{enumerate}[label = \textbullet]
  \item \(1\) identity.
  \item \(3\) rotations by \(90^\circ\) through an axis through midpoints of opposite faces.
  \item \(3\) rotations by \(180^\circ\) through an axis through midpoints of opposite faces.
  \item \(3\) rotations by \(270^\circ\) through an axis through midpoints of opposite faces.
  \item \(6\) rotations by \(180^\circ\) through an axis through midpoints of opposite edges.
  \item \(4\) rotations by \(120^\circ\) through an axis through opposite vertices.
  \item \(4\) rotations by \(240^\circ\) through an axis through opposite vertices.
\end{enumerate}
