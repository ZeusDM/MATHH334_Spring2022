\section{Burnside's Lemma}
\lecture{8}{2022-02-04}{Burnside's Lemma}

\begin{defn}{Fixed set}{fixedset}
  Suppose \(G\) acts on \(S\). Let \(g \in G\). The fixed set of \(g\) is \[
    S^g = \{s \in S : g*s = s\}.
  \] 
\end{defn}

\begin{thm}{Burnside's Lemma}{burnside}
  Suppose a finite group \(G\) acts on \(S\). Then, \[
    \#(\text{orbits}) = \frac{1}{|G|} \sum_{g \in G} |S^g|,
  \] 
  i.e., the number of orbits is the average size of the fixed sets.
\end{thm}

\begin{dem}{}{}
  Note that
  \begin{align*}
	\#(\text{orbits}) &= \sum_{s \in S} \frac{1}{\left|\mathcal{O}_s\right|} \\
					  &= \sum_{s \in S} \frac{\left|G_s\right|}{\left|G\right|} \\
					  &= \frac{1}{\left|G\right|} \sum_{s \in S} \left|G_s\right| \\
                      &= \frac{1}{\left|G\right|} \sum_{g \in G} \left|S^g\right| & \text{(by double counting)}
  \end{align*}
\end{dem}

\subsection*{Application}

\begin{que}{}{}
A board game piece is a cube, in which each face is colored in one of three colors.
Thus, each piece is associated with a \(3\)-coloration of the \(6\) faces of the cube. 

We say that two board game pieces are indistinguishable when one can be rotated into the other. What is the maximum number of distinguishable game pieces?
\end{que}

Let \(G\) be the group of rotations of the cube.
Let \(S\) be the set of \(3\)-colorations of the faces of the cube.
Let \(G\) act on \(S\) naturally.

The answer to the question is the number of orbits of this action; since distinguishable colorations must be in distinct orbits.

There are some types of rotations:
\begin{enumerate}[label = \textbullet]
  \item \(1\) identity.
  \item \(3\) rotations by \(90^\circ\) through an axis through midpoints of opposite faces.
  \item \(3\) rotations by \(180^\circ\) through an axis through midpoints of opposite faces.
  \item \(3\) rotations by \(270^\circ\) through an axis through midpoints of opposite faces.
  \item \(6\) rotations by \(180^\circ\) through an axis through midpoints of opposite edges.
  \item \(4\) rotations by \(120^\circ\) through an axis through opposite vertices.
  \item \(4\) rotations by \(240^\circ\) through an axis through opposite vertices.
\end{enumerate}

Respectively, their fixed sets have \(3^6\), \(3^3\), \(3^4\), \(3^3\), \(3^3\), \(3^2\), \(3^2\), elements.
Finally, by \nameref{thm:burnside}, we conclude that the number of orbits is \[
  \#(\text{orbits}) = \frac{1}{24} \left(
    1 \cdot 3^6 +
    3 \cdot 3^3 +
    3 \cdot 3^4 +
    3 \cdot 3^3 +
    6 \cdot 3^3 +
    4 \cdot 3^2 +
    4 \cdot 3^2\right)
    = \frac{1368}{24} = 57.
\] 


