\lecture{13}{2022-02-16}{}

\begin{defn}{Character of a Representation}{}
	Given a representation \(\rho\colon G \to GL_n(\mathbb{C})\), let \(\chi_\rho\colon G \to \mathbb{C}\) be defined by \(\chi_\rho = \Tr \circ \rho\).
\end{defn}

If a representation is indexed, e.g. \(\rho_T\), then we usually denote its character by the same index, e.g. \(\chi_T = \chi_{\rho_T}\).

\begin{exmp}{Trivial Representation}{}
	Let \(G\) be any group. The map \(\rho_T \colon G \to GL_1(\mathbb{C}) = \mathbb{C}^\times\) defined by \[
		\rho_T(g) = 1
	\] for all \(g \in G\), is the \emph{trivial representation} of \(G\).
\end{exmp}

\begin{exmp}{Permutation Representation}{}
	The map \(\rho_P \colon S_n \to GL_n(\mathbb{C})\) defined by \[
		\rho_T(\pi) = 
		\begin{bmatrix}
			\mathbf{e}_{\pi(1)} &
			\mathbf{e}_{\pi(2)} &
			\cdots &
			\mathbf{e}_{\pi(n)} 
		\end{bmatrix}
	\] for all \(\pi \in S_n\), is the permutation representation.
\end{exmp}

\begin{exmp}{(Left) Regular Representation}{}
	Consider the action of \(G\) on itself defined by \(g * h = gh\). This yields a representation along the lines of Proposition \ref{prop:action-representation}, with \(S = G\).
\end{exmp}

\begin{exmp}{Sign Representation of \(S_n\)}{}
	The map \(\rho_S \colon S_n \to GL_1(\mathbb{C}) = \mathbb{C}^\times\) defined by \[
		\rho_S(\pi) = 
		\begin{cases}
			1 & \text{if \(\pi\) is an even permutation,} \\
			-1 & \text{otherwise,}
		\end{cases}
	\] for all \(\pi \in S_n\), is the \emph{sign representation} of \(S_n\).
\end{exmp}

\begin{exmp}{Liz's Standard Representation}{}
  Suppose \(G\) act on \(\mathbb{C}^n\), and \(\pi_g \colon \mathbb{C}^n \to \mathbb{C}^n\) defined by \(z \mapsto g * z\) is a linear trasformations. The \emph{standard representation} \(\rho_A \colon G \to GL_n(\mathbb{C})\) is defined by \[
    g \mapsto 
	\begin{bmatrix}
	  g * \mathbf{e}_1 &
	  \cdots &
	  g * \mathbf{e}_n
	\end{bmatrix}
  \] 
\end{exmp}
