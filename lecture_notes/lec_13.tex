\lecture{13}{2022-02-16}{}

\begin{defn}{Character of a Representation}{}
	Given a representation \(\rho\colon G \to GL_n(\mathbb{C})\), let \(\chi_\rho\colon G \to \mathbb{C}\) be defined by \(\chi_\rho = \Tr \circ \rho\).
\end{defn}

If a representation is indexed, e.g. \(\rho_T\), then we usually denote its character by the same index, e.g. \(\chi_T = \chi_{\rho_T}\).

\begin{exmp}{Trivial Representation}{}
  
\end{exmp}

\begin{exmp}{Permutation Representation}{}
  
\end{exmp}

\begin{exmp}{(Left) Regular Representation}{}
  
\end{exmp}

\begin{exmp}{Sign Representation of \(S_n\)}{}
  
\end{exmp}

\begin{exmp}{Liz's Standard Representation}{}
  Suppose \(G\) act on \(\mathbb{C}^n\), and \(\pi_g \colon \mathbb{C}^n \to \mathbb{C}^n\) defined by \(z \mapsto g * z\) is a linear trasformations. The \emph{standard representation} \(\rho_A \colon G \to GL_n(\mathbb{C})\) is defined by \[
    g \mapsto 
	\begin{bmatrix}
	  g * \mathbf{e}_1 &
	  \cdots &
	  g * \mathbf{e}_n
	\end{bmatrix}
  \] 
\end{exmp}
