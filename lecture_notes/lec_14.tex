\lecture{14}{2022-02-18}{}

\begin{exmp}{Observation}{}
	If \(G\) is generated by \(S\), and \(\rho\colon G \to GL_n(\mathbb{C})\) is an representation; then we can just calculate \(\rho(s)\) for \(s \in S\), and let a computer calculate the others by doing matrix multiplications.
\end{exmp}

\begin{exmp}{}{}
	Consider \(\rho_\textrm{3D}, \rho_\textrm{2D}, \rho_\textrm{sign} \colon D_3 \to GL_n(\mathbb{C})\). Then, \[
		\rho_\textrm{3D} = \rho_\textrm{2D} \oplus \rho_\textrm{sign}.
	\] 

	Magically, we also have \[
		\chi_\textrm{3D} = \chi_\textrm{2D} + \chi_\textrm{sign}.
	\] 
\end{exmp}

\begin{defn}{\(G\)-invariant Subspace}{}
	Suppose \(G\) acts on \(\mathbb{C}^n\) by linear transformations, then a vector subspace \(V \subset \mathbb{C}^n\) is \emph{\(G\)-invariant} if \[
		\rho(g) \mathbf{v} \in V
	\] 
	for all \(g \in G\) and all \(\mathbf{v} \in V\).
\end{defn}

\begin{defn}{Irreducible Representation}{}
	A representation of \(G\) is called \emph{irreducible} if it has no proper \(G\)-invariant subspace.
\end{defn}
