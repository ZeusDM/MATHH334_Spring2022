\lecture{21}{2022-03-14}{}

\chapter{Moving Forward}

\begin{defn}{Equivalence}{}
	Two representations \(\rho_1 \colon G \to GL_n(\mathbb{C})\) and \(\rho_2 \colon G \to GL_n(\mathbb{C})\) are said to be \emph{equivalent}, denoted by \(\rho_1 \cong \rho_2\), if there exists a general linear transformation \(T \in GL_n(\mathbb{C})\) such that \[
		\rho_2(g) = T \rho_1(g) T^{-1},
	\] 
	for all \(g \in G\).
\end{defn}

\section{Next Big Goal}

\begin{thm}{Maschke's Theorem}{maschke}
	Let \(G\) be a finite group. Every representation \(\rho \colon \mathbb{R} \to GL_n(\mathbb{C})\) satisfies \[
		\rho \cong p_1 \oplus p_2 \oplus \cdots \oplus \cdots \rho_m.
	\] 
\end{thm}
