\section{Orbits}
\lecture{3}{2022-01-24}{Orbits and Stabilizers}

\begin{defn}{Orbits}{orbits}
  Suppose \(G\) acts on \(S\). Fix \(s \in S\). The \emph{orbit of \(s\)} is
  \[
    \mathcal{O}_s = \{g * s : g \in G\}.
  \] 
\end{defn}

\begin{exmp}{}{}
  Let \(S = \mathbb{R}^2\) and \(G\) be the group of rotations of \(\mathbb{R}\) around the origin.
  We denote by \(\rho_\theta \in G\) the rotation around the origin by \(\theta\).
  Define \(\rho_\theta * \vec x = \rho_\theta(\vec x)\).

  If we fix \(\vec x \in \mathbb{R}^2\), the orbit \(\mathcal{O}_{\vec x}\) is the circle centered at the origin with radius  \(\left|\vec x\right|\).
\end{exmp}

\begin{exmp}{}{}
	Let \(G\) be a group. Define the group action \emph{left multiplication} of \(G\) on itself defined by \(g * s = gs\).

    If we fix \(x \in G\), note that \((yx^{-1})*x=yx^{-1}x=y\); therefore the orbit \(\mathcal{O}_x\) is \(G\).
\end{exmp}

\section{Stabilizers}

\begin{defn}{Stabilizer}{}
  Suppose \(G\) acts on \(S\). Fix \(s \in S\). The \emph{stabilizer of \(s\)} is
  \[
    G_s = \{g \in G : g * s = s\}.
  \] 
\end{defn}

\begin{exmp}{}{}
  Let \(S = \mathbb{R}^2\) and \(G\) be the group of rotations of \(\mathbb{R}\) around the origin.
  We denote by \(\rho_\theta \in G\) the rotation around the origin by \(\theta\).
  Define \(\rho_\theta * \vec x = \rho_\theta(\vec x)\).

  If we fix \(\vec x \neq \vec 0 \in \mathbb{R}^2\), the stabilizer \(S_{\vec x}\) is \(\{\rho_0\}\).
  The stabilizer \(S_{\vec 0}\) is \(G\).
\end{exmp}

\begin{exmp}{}{}
	Let \(G\) be a group. Define the group action \emph{left multiplication} of \(G\) on itself defined by \(g * s = gs\).

    If we fix \(x \in G\), note that \((gx=x \iff g=e\); therefore the orbit \(\mathcal{O}_x\) is \(\{e\}\).
\end{exmp}


